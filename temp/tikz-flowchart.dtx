% \iffalse meta-comment
%
% Copyright (C) 2019 by Geng Nan <nangeng@nwsuaf.edu.cn>
%
% This file may be distributed and/or modified under the
% conditions of the LaTeX Project Public License, either
% version 1.3 of this license or (at your option) any later
% version.  The latest version of this license is in:
%
%    http://www.latex-project.org/lppl.txt
%
% and version 1.3 or later is part of all distributions of
% LaTeX version 2005/12/01 or later.
%
% \fi
%
% \iffalse
%<package>\NeedsTeXFormat{LaTeX2e}[2011/06/27]
%<package>\ProvidesPackage{tikz-flowchart}
%<package>  [2019/08/20 v1.0.01 draw flowchart using TikZ]
%
%<*driver>
\documentclass[a4paper]{ltxdoc}
\RecordChanges

\usepackage{ctex}
\usepackage{lmodern}
\usepackage{hyperref}
\usepackage{url}
\usepackage{amsmath}
\usepackage{amssymb}
\usepackage{tikz-flowchart}
\usepackage{float}
%\usepackage{parskip}

% default position for floats: H
\makeatletter
\renewcommand{\fps@figure}{H}
\renewcommand{\fps@table}{H}
\makeatother

%\setlength{\parindent}{0pt}

\def\pkg{\texttt{tikz-flowchart}}
\def\tkz{Ti\emph{k}Z}

\begin{document}
  \DocInput{tikz-flowchart.dtx}
\end{document}
%</driver>
% \fi
%
% \CheckSum{0}
%
% \CharacterTable
%  {Upper-case    \A\B\C\D\E\F\G\H\I\J\K\L\M\N\O\P\Q\R\S\T\U\V\W\X\Y\Z
%   Lower-case    \a\b\c\d\e\f\g\h\i\j\k\l\m\n\o\p\q\r\s\t\u\v\w\x\y\z
%   Digits        \0\1\2\3\4\5\6\7\8\9
%   Exclamation   \!     Double quote  \"     Hash (number) \#
%   Dollar        \$     Percent       \%     Ampersand     \&
%   Acute accent  \'     Left paren    \(     Right paren   \)
%   Asterisk      \*     Plus          \+     Comma         \,
%   Minus         \-     Point         \.     Solidus       \/
%   Colon         \:     Semicolon     \;     Less than     \<
%   Equals        \=     Greater than  \>     Question mark \?
%   Commercial at \@     Left bracket  \[     Backslash     \\
%   Right bracket \]     Circumflex    \^     Underscore    \_
%   Grave accent  \`     Left brace    \{     Vertical bar  \|
%   Right brace   \}     Tilde         \~}
%
% \changes{v1.0.01}{2019/08/20}{chang .sty file to .dtx file}
%
% \GetFileInfo{tikz-flowchart.sty}
%
% \title{\texttt{tikz-flowchart}---\tkz 流程图绘制宏包\thanks{该文档是 \texttt{tikz-flowchart}~\fileversion, dated~\filedate 的说明文档。}}
% \author{耿楠\thanks{https://github.com/registor/tikz-flowchart}\\ 西北农林科技大学信息工程学院计算机科学系}
% \date{\filedate}
% \thispagestyle{empty}
% \maketitle
%
% \begin{abstract}
%   这是一个使用\tkz 绘制传统程序流程图的简单宏包,通过定义\texttt{proc}、\texttt{test}、\texttt{io}、\texttt{term}等\tkz 的\texttt{node}命令样式实现。
% 该宏包核心代码摘录自\href{http://www.texample.net/tikz/examples/author/brent-longborough/}{Brent Longborough}设计的流程图绘制样例,
% 参考了\texttt{tikz-imagelabels}宏包的设计思路,提供了\texttt{\textbackslash flowchartset}命令以设置绘制参数。
% \end{abstract}
%
% \tableofcontents
%
% \PrintChanges
%
% \StopEventually{}
%
% \section{宏包简介}
%
% 流程图是诸如手册、报告、论文等文档中经常用到的排版元素,\pkg{}宏包的目的是为了更为方便地实现传统流程图的绘制。
%
% \clearpage
% \autoref{fig:pleiades}由如下代码绘制:
% \begin{verbatim}
% \begin{annotationimage}{width=6cm}{pleiades.jpg}
%   \draw[annotation left = {Atlas at 0.3}] to (0.11,0.4);
%   \draw[annotation left = {Pleione at 0.55}] to (0.11,0.49);
%   \draw[annotation left = {Alcyone at 0.8}] to (0.39,0.45);
%   \draw[annotation below = {Merope at 0.5}] to (0.58,0.28);
%   \draw[annotation right = {Electra at 0.3}] to (0.84,0.45);
%   \draw[annotation right = {Caleano at 0.75}] to (0.85,0.64);
%   \draw[annotation above = {Maia at 0.4}] to (0.67,0.72);
%   \draw[annotation above = {Taygeta at 0.9}] to (0.78,0.82);
%   \draw[image label = {M45 at south east}];
% \end{annotationimage}
% \end{verbatim}
%
% \section{使用方法}
%
% \subsection{布置结点}
%
% \DescribeEnv{annotationimage}
% To include an image, the |annotationimage| environment is used. It
% has the following syntax:
%
% |annotationimage|\oarg{grid}\marg{options}\marg{file name}
%
% The \meta{grid} is an optional parameter. If this parameter is present, i.e.
% if it has the value |[grid]|, then a coordinate grid is visible. The
% coordinate grid is used to find the coordinates of the points to be labelled.
% If the parameter \meta{grid} is omitted, no coordinate grid is drawn. The code
%
% \begin{verbatim}
% \begin{annotationimage}[grid]{width=6cm}{pleiades.jpg}
% \end{annotationimage}
% \end{verbatim}
%
% produces the image shown in \autoref{fig:grid-example}.
%
% \meta{options} is any set of options understood by the
% |\includegraphics| command, e.g. |width=|, |height=| and so on. It may also be
% left empty, but in this case, the curly braces need to be there, though.
%
% The \meta{file name} is, obviously, the file name of the image. Like for the
% \meta{options}, any image format supported by |\includegraphics| may be used.
%
% \subsection{Adding a label}
%
% A label (like the ``M45'' in \autoref{fig:pleiades}) can be added to the image
% using following |\draw| macro:
%
% |\draw[image label = {|\meta{text}| at |\meta{placement}|}];|
%
% The \meta{text} parameter is obvious. It contains the text to be put into the
% label.
%
% The \meta{placement} dictates the placement of the label. It may be one of
% |north west|, |north|, |north east|, |east|, |south east|, |south|,
% |south west| or |west|. Also |center| is possible, even though it possibly
% doesn't make a lot of sense. \autoref{fig:label_example} shows an example
% with several labels.
%
% The label in \autoref{fig:pleiades} was drawn using the following code:
%
% |\draw[image label = {M45 at south east}];|
%
% Another kind of label is the so-called ``coordinate label''. In contrast to
% the image label, it may be positioned at arbitrary coordinates. The syntax is
% similar to the |image label|:
%
% |\draw[coordinate label = {|\meta{text}| at (|\meta{coordinate}|)}];|
%
% An example of some coordinate labels is shown in \autoref{fig:coord-labels}.
%
%
% \autoref{fig:coord-labels} was created using the following code:
% \begin{verbatim}
%   \draw[coordinate label = {1 at (0.1,0.3)}];
%   \draw[coordinate label = {2 at (0.4,0.6)}];
%   \draw[coordinate label = {3 at (0.65,0.25)}];
%   \draw[coordinate label = {4 at (0.9,0.4)}];
%   \draw[coordinate label = {5 at (0.62,0.62)}];
%   \draw[coordinate label = {6 at (0.83,0.8)}];
%   \draw[image label = {M45 at south east}];
% \end{verbatim}
%
% \subsection{Adding annotations}
%
% An annotation is added with the aid of the \tkz{} macro |\draw|. The syntax is
% as follows:
%
% |\draw[annotation |\meta{placement}| = {|\meta{text}| at |\meta{position}|}] to (|\meta{x}|, |\meta{y}|);|
%
% The \meta{placement} is one of: |above|, |right|, |below| or |left|. It tells
% on which side of the image the annotation will appear. \meta{above} and
% \meta{below} basically determine the $y$ coordinate of the text, while
% \meta{left} and \meta{right} determine the $x$ coordinate of the text. The
% remaining coordinate is determined using the \meta{position}.
%
% The \meta{text} is the actual text of the annotation.
%
% The \meta{x} and \meta{y} parameters are the actual coordinates where the
% arrow should point to. Note that \pkg{} will automatically insert a small
% distance between the arrow's tip and the coordinate given, such that the
% arrow is close to the coordinate but does not cover it.
%
% For example, the code
%
% |\draw[annotation left = {Atlas at 0.3}] to (0.11,0.4);|
%
% draws the text ``Atlas'' on the left-hand side of the image, at $y=0.3$. The
% arrow will point towards coordinate $(0.11, 0.4)$ but ends shortly before this
% coordinate such that the interesting feature to be labelled is not covered by
% the arrow's tip.
%
% \section{Configuring styles}
%
% \DescribeMacro{\flowchartset}
% Various options, like font size and so on, can be configured with the
% |\flowchartset| macro. It uses the key-value syntax from \tkz{}, e.g.:
%
% |\flowchartset{|\meta{key}| = |\meta{value}|, ...}|
%
% Multiple \meta{key} and \meta{value} pairs may be combined. The following
% sections list all possible configurations.
%
% |\flowchartset| can be put anywhere, but it makes sense to put it into the
% preamble of a document to ensure all images have the same look.
%
% There is also a default style. If no |\flowchartset| command is present, the
% default values for all options are taken. The default style used is as
% follows:
%
% \begin{verbatim}
% \flowchartset{
%   coarse grid color = red,
%   fine grid color = gray,
%   image label font = \sffamily\bfseries\small,
%   image label distance = 2mm,
%   image label back = black,
%   image label text = white,
%   coordinate label font = \sffamily\bfseries\scriptsize,
%   coordinate label distance = 2mm,
%   coordinate label back = black,
%   coordinate label text = white,
%   annotation font = \normalfont\small,
%   arrow distance = 1.5mm,
%   border thickness = 0.6pt,
%   arrow thickness = 0.4pt,
%   tip size = 1.2mm,
%   outer dist = 0.5cm,
% }
% \end{verbatim}
%
% The individual keys are described in the following sections.
%
% \subsection{Grid color}
%
% In most cases, it will not be necessary to adjust the grid colors. However,
% depending on the image, it may be desirable to do so. This is exactly what the
% \meta{coarse grid color} and \meta{fine grid color} are used for. Any color
% specification compatible to \tkz{} may be used. The defaults are red for the
% coarse grid and gray for the fine grid.
%
% \subsection{Font and color for the labels}
%
% The font for the image labels may be configured with \meta{image label font}.
% By default, the image labels are typeset with bold, small, sans-serif font.
%
% The background color of the image labels may be set using the
% \meta{image label back} kay, whereas the text color is specified with the
% \meta{image label text} key. Defaults for the background color and for the
% text color are black and white, respectively.
%
% \subsection{Distance of image labels to the image border}
%
% The \meta{image label distance} key configures the distance, $d$, from the
% image's border to the border of the image label, as shown in
% \autoref{fig:imagelabeldistance_example}. By default, it is set to 2\,mm.
%
% \begin{figure}
% \centering
% \flowchartset{
%    image label distance = 1cm}
% \begin{annotationimage}{width=5cm}{example-image}
%   \draw[image label = {Label at south east}];
%   \draw[>=latex, red, <->] (1,0.3) -- ++(-1cm,0) node[above, midway] {$d$};
%   \draw[>=latex, red, <->] (0.7,0) -- ++(0,1cm) node[midway, right] {$d$};
% \end{annotationimage}
% \caption{Illustration of the \meta{image label distance}}
% \label{fig:imagelabeldistance_example}
% \end{figure}
%
% \subsection{Font for annotations}
%
% The font used for annotations is set by \meta{annotation font}. By default,
% the |\normalfont| is used with small size.
%
% \subsection{Distance of arrow tips}
%
% As mentioned earlier, the arrows are shortened such that their tips don't
% cover the desired point. \autoref{fig:arrowdistance_example} illustrates this.
% All the arrows point to the same coordinate, $(0.5, 0.5)$, but they end at
% the distance $x$ away from the point. This distance may be configured using
% the \meta{arrow distance}. By default, this distance is set to 1.5\,mm. This
% ensures that the arrow tips are close enough to the interesting features, but
% not so close that they cover important parts of the image.
%
% \begin{figure}
% \centering
% \flowchartset{arrow distance = 1cm}
% \begin{annotationimage}{width=5cm}{example-image}
%   \draw[annotation right = {text at 0.5}] to (0.5,0.5);
%   \draw[annotation left = {text at 0.1}] to (0.5,0.5);
%   \draw[annotation left = {text at 0.7}] to (0.5,0.5);
%   \draw[annotation below = {text at 0.4}] to (0.5,0.5);
%   \draw[annotation above = {text at 0.8}] to (0.5,0.5);
%   \draw[red] (0.5,0.5) circle[radius = 1cm];
%   \draw[>=latex, red, <->] (0.5,0.5) -- ++(60:1cm) node[midway, left] {$x$};
% \end{annotationimage}
% \caption{Illustration of the \meta{arrow distance}}
% \label{fig:arrowdistance_example}
% \end{figure}
%
% \subsection{Thickness and size of the arrows}
%
% The arrows themselves consist of two parts: the inner part, which is the
% actual arrow, and the border, which is, by default, a white border around the
% arrow. The border is required to ensure that each arrow is visible, no matter
% on what background it is drawn. The thickness of the black line can be
% configured using the \meta{arrow thickness}, whose default value is 0.4\,pt.
% The thickness of the border around the arrow is configured with the
% \meta{border thickness}, having a default value of 0.6\,pt.
%
% The size of the round dot at the end of the arrows is configured using the
% \meta{tip size}. \autoref{fig:arrowthickness_tipsize_example} illustrates
% both, the \meta{arrow thickness}, and the \meta{tip size}, as parameters
% $a$ and $b$, respectively.
%
% \begin{figure}
% \centering
% \flowchartset{arrow thickness = 1cm, tip size = 2cm, arrow distance=0}
%   \let\tikzset\flowchartset
% \begin{tikzpicture}
%   \draw[annotation arrow] (0,0) -- (4cm,0);
%   \draw[>=latex, red, <->] (0.5,-0.5) -- ++(0,1cm) node[right, midway] {$a$};
%   \draw[>=latex, red, <->] (3cm,-1cm) -- ++(0,2cm) node[right, midway] {$b$};
% \end{tikzpicture}
% \caption{Illustration of the \meta{arrow thickness}, $a$, and
% the \meta{tip size}, $b$}
% \label{fig:arrowthickness_tipsize_example}
% \end{figure}
%
% \subsection{Distance of annotation texts from the image}
%
% The parameter \meta{outer dist} configures how far away from the image the
% annotation texts will be positioned. By default, this value is 0.5\,cm.
%
% \section{Implementation}
% The only packages required are |tikz| and |xifthen|. If not already loaded,
% they will be loaded automatically.
%    \begin{macrocode}
\RequirePackage{tikz}
\RequirePackage{xifthen}
%    \end{macrocode}
% Some \tkz{} libraries are also reuiqred for proper operation.
%    \begin{macrocode}
\usetikzlibrary{
  arrows.meta,
  calc,
  positioning,
  decorations,
  decorations.markings,
  math,
}
%    \end{macrocode}
% \subsection{Configuration}
% For the |\flowchartset| command, a |pgfkeys| family is defined. All
% configurations (e.g. \meta{tip size} and so on) and styles are stored in the
% PGF key |/flowchart|. This ensures that these configurations don't overwrite
% any other parameters the user may have set elsewhere.
%    \begin{macrocode}
\pgfkeys{
  /flowchart/.is family,
  /flowchart/.search also={/tikz},
}

\def\flowchartset{\pgfqkeys{/flowchart}}
%    \end{macrocode}
% Then, a set of macros is created which stores the values for the individual
% configuration options.
%    \begin{macrocode}
\flowchartset{
  coarse grid color/.store in = \maingridcolor,
  fine grid color/.store in = \finegridcolor,
}

\flowchartset{
  image label font/.store in = \imagelabelfont,
  image label distance/.store in = \flowchartep,
  image label back/.store in = \imagelabelback,
  image label text/.store in = \imagelabeltext,
}

\flowchartset{
  coordinate label font/.store in = \coordinatelabelfont,
  coordinate label distance/.store in = \coordinatelabelsep,
  coordinate label back/.store in = \coordinatelabelback,
  coordinate label text/.store in = \coordinatelabeltext,
}

\flowchartset{
  annotation font/.store in = \annotationfont,
  arrow distance/.store in = \arrowdistance,
  arrow thickness/.store in = \arrowthickness,
  tip size/.store in = \tipsize,
  border thickness/.store in = \borderthickness,
  outer dist/.store in = \labeloutersep,
}
%    \end{macrocode}
%
% \subsection{Default configuration}
% The default configuration is set. This will ensure that each of the previously
% defined macros has a valid initial value, which may be overwritten by the
% user.
%    \begin{macrocode}
\flowchartset{
  coarse grid color = red,
  fine grid color = gray,
  image label font = \sffamily\bfseries\small,
  image label distance = 2mm,
  image label back = black,
  image label text = white,
  coordinate label font = \sffamily\bfseries\scriptsize,
  coordinate label distance = 2mm,
  coordinate label back = black,
  coordinate label text = white,
  annotation font = \normalfont\small,
  arrow distance = 1.5mm,
  border thickness = 0.6pt,
  arrow thickness = 0.4pt,
  tip size = 1.2mm,
  outer dist = 0.5cm,
}
%    \end{macrocode}
%
% \subsection{Environment declaration}
% Next, the |annotationimage| environment is declared. It takes 3 arguments,
% the first of which is optional. If it is omitted, its default value is empty.
%    \begin{macrocode}
\newenvironment{annotationimage}[3][]{
%    \end{macrocode}
% Each time a new |annotationimage| environment is opened, this code will ensure
% that all the definitions stored under the PGF key |/flowchart| are loaded.
% Then, a new |tikzpicture| is created.
%    \begin{macrocode}
\let\tikzset\flowchartset
  \begin{tikzpicture}
%    \end{macrocode}
% The 2nd and 3rd arguments to the |annotationimage| are the size/scaling
% options for the image, as well as the actual image file. Thus, a new \tkz{}
% node called |image| is created; the node's content is the image.
%    \begin{macrocode}
  \node[anchor=south west, inner sep=0]
    (image) at (0,0) {\includegraphics[#2]{#3}};
%    \end{macrocode}
% Using a scope ensures that the top-right corner always has coordinate $(1,1)$.
%    \begin{macrocode}
  \begin{scope}[x={(image.south east)},y={(image.north west)}]
%    \end{macrocode}
% Next, the first (optional) argument's value is checked. If the user said
% |[grid]| to the first argument, the following code is executed.
%    \begin{macrocode}
  \ifthenelse{\equal{#1}{grid}}{%
%    \end{macrocode}
% This actually draws the coordinate grid.
%    \begin{macrocode}
    \draw[very thin, draw=\finegridcolor, step=0.02]
      (0,0) grid (1,1);
    \draw[thin, draw=\maingridcolor, xstep=0.1, ystep=0.1]
      (0,0) grid (1,1);
%    \end{macrocode}
% then, the labels are put to the coordinate axes.
%    \begin{macrocode}
    \foreach \x in {0,1,...,9} {
      \node [anchor=north] at (\x/10,0) {\tiny 0.\x};
    }
    \node [anchor=north] at (1,0) {\tiny 1};

    \foreach \y in {0,1,...,9} {
      \node [anchor=east] at (0,\y/10) {\tiny 0.\y};
    }
    \node [anchor=east] at (0,1) {\tiny 1};
  }{}
}
{
%    \end{macrocode}
% Each time the |annotationimage| environment is closed, the previously opened
% |scope| and |tikzpicture| environments need to be closed as well.
%    \begin{macrocode}
 \end{scope}
 \end{tikzpicture}}
%    \end{macrocode}
%
% \subsection{Style definitions for the annotations}
% What follows is the definition of the styling for the annotations.
%    \begin{macrocode}
\flowchartset{
%    \end{macrocode}
% This is the style for the annotation arrow itself.
%    \begin{macrocode}
  annotation arrow/.style =
  {
%    \end{macrocode}
% The |preaction| first draws a thick white arrow. This arrow will become the
% border.
%    \begin{macrocode}
    preaction =
    {
      draw,
      -{Circle[fill=white, length=\tipsize+2*\borderthickness,
        width=\tipsize+2*\borderthickness]},
      line width = 2*\borderthickness + \arrowthickness,
      white,
      shorten >= \arrowdistance,
    },
%    \end{macrocode}
% After the |preaction| has been performed, this will actually draw the
% ``normal'' arrow.
%    \begin{macrocode}
    draw,
    -{Circle[fill=black, length=\tipsize, width=\tipsize]},
    black,
    line width = \arrowthickness,
    shorten >= \borderthickness + \arrowdistance,
  },
%    \end{macrocode}
% All the annotation texts have a common style. This ensures they have the same
% font etc. Setting the |inner sep| to 0.5\,ex ensures that the distance between
% the text and the arrow is somewhat aesthetic. It is an empirically determined
% value.
%    \begin{macrocode}
  annotation node/.style =
  {
    font=\annotationfont,
    inner sep = 0.5ex,
  },
%    \end{macrocode}
% Next comes the styles for the different annotation placements. For an
% annotation being below the image, this style applies.
%    \begin{macrocode}
  annotation below/.style args = {#1 at #2}{
%    \end{macrocode}
% Using the annotation arrow style tells \tkz{} to draw an arrow as specified
% above, using the geometry defined with |\flowchartset|.
%    \begin{macrocode}
    annotation arrow,
%    \end{macrocode}
% After the arrow has been drawn, a further path is inserted, which is the
% actual annotation text. For the annotations being above and below the image
% special care must be taken: a |\strut| is appended to the label text to ensure
% that texts being on the same side of the image are on the same line. Without
% the strut, the texts may be differently aligned, depending on their letters --
% e.g. letters ``p'' and ``g'' go slightly further down in the $y$ direction
% than ``a'' or ``b''.
%    \begin{macrocode}
    insert path = {
      (#2,0) ++ (0,-\labeloutersep)
        node[anchor = north, annotation node] {#1\strut}
    }
  },
%    \end{macrocode}
% The remaining annotation styles are defined similarly.
%    \begin{macrocode}
  annotation above/.style args = {#1 at #2}{
    annotation arrow,
    insert path = {
      (#2,1.0) ++ (0,\labeloutersep)
        node[anchor = south, annotation node] {#1\strut}
    }
  },
  annotation left/.style args = {#1 at #2}{
    annotation arrow,
    insert path = {
      (0,#2) ++ (-\labeloutersep,0)
        node[anchor = east, annotation node] {#1}
    }
  },
  annotation right/.style args = {#1 at #2}{
    annotation arrow,
    insert path = {
      (1.0,#2) ++ (\labeloutersep,0)
        node[anchor = west, annotation node] {#1}
    }
  },
}
%    \end{macrocode}
%
% \subsection{Style definitions for the labels}
% Next follows the style definition for the image labels. A general style
% defines the appearance and color.
%    \begin{macrocode}
\flowchartset{
  image label style/.style = {
    rectangle,
    minimum width = 5mm,
    minimum height = 5mm,
    fill = \imagelabelback,
    text = \imagelabeltext,
    font = \imagelabelfont,
  },
  coordinate label style/.style = {
    rectangle,
    minimum width = 3mm,
    minimum height = 3mm,
    fill = \coordinatelabelback,
    text = \coordinatelabeltext,
    font = \coordinatelabelfont,
  },
%    \end{macrocode}
% On the other hand, the |image label| style defines the actual image labels.
%    \begin{macrocode}
  image label/.style args = {#1 at #2}{
    insert path = {
      (image.#2) node[outer sep = \flowchartep,
        anchor=#2, image label style] {#1}
    }
  },
  coordinate label/.style args = {#1 at (#2)}{
    insert path = {
      node[outer sep = \coordinatelabelsep,
        anchor=center, coordinate label style] at (#2) {#1}
    }
  },
}
%    \end{macrocode}
%
% \Finale
\endinput
