%
% \DescribeEnv{annotationimage}
% To include an image, the |annotationimage| environment is used. It
% has the following syntax:
%
% |annotationimage|\oarg{grid}\marg{options}\marg{file name}
%
% The \meta{grid} is an optional parameter. If this parameter is present, i.e.
% if it has the value |[grid]|, then a coordinate grid is visible. The
% coordinate grid is used to find the coordinates of the points to be labelled.
% If the parameter \meta{grid} is omitted, no coordinate grid is drawn. The code
%
% \begin{verbatim}
% \begin{annotationimage}[grid]{width=6cm}{pleiades.jpg}
% \end{annotationimage}
% \end{verbatim}
%
% produces the image shown in \autoref{fig:grid-example}.
%
% \meta{options} is any set of options understood by the
% |\includegraphics| command, e.g. |width=|, |height=| and so on. It may also be
% left empty, but in this case, the curly braces need to be there, though.
%
% The \meta{file name} is, obviously, the file name of the image. Like for the
% \meta{options}, any image format supported by |\includegraphics| may be used.
%
% \subsection{Adding a label}
%
% A label (like the ``M45'' in \autoref{fig:pleiades}) can be added to the image
% using following |\draw| macro:
%
% |\draw[image label = {|\meta{text}| at |\meta{placement}|}];|
%
% The \meta{text} parameter is obvious. It contains the text to be put into the
% label.
%
% The \meta{placement} dictates the placement of the label. It may be one of
% |north west|, |north|, |north east|, |east|, |south east|, |south|,
% |south west| or |west|. Also |center| is possible, even though it possibly
% doesn't make a lot of sense. \autoref{fig:label_example} shows an example
% with several labels.
%
% The label in \autoref{fig:pleiades} was drawn using the following code:
%
% |\draw[image label = {M45 at south east}];|
%
% Another kind of label is the so-called ``coordinate label''. In contrast to
% the image label, it may be positioned at arbitrary coordinates. The syntax is
% similar to the |image label|:
%
% |\draw[coordinate label = {|\meta{text}| at (|\meta{coordinate}|)}];|
%
% An example of some coordinate labels is shown in \autoref{fig:coord-labels}.
%
%
% \autoref{fig:coord-labels} was created using the following code:
% \begin{verbatim}
%   \draw[coordinate label = {1 at (0.1,0.3)}];
%   \draw[coordinate label = {2 at (0.4,0.6)}];
%   \draw[coordinate label = {3 at (0.65,0.25)}];
%   \draw[coordinate label = {4 at (0.9,0.4)}];
%   \draw[coordinate label = {5 at (0.62,0.62)}];
%   \draw[coordinate label = {6 at (0.83,0.8)}];
%   \draw[image label = {M45 at south east}];
% \end{verbatim}
%
% \subsection{Adding annotations}
%
% An annotation is added with the aid of the \tkz{} macro |\draw|. The syntax is
% as follows:
%
% |\draw[annotation |\meta{placement}| = {|\meta{text}| at |\meta{position}|}] to (|\meta{x}|, |\meta{y}|);|
%
% The \meta{placement} is one of: |above|, |right|, |below| or |left|. It tells
% on which side of the image the annotation will appear. \meta{above} and
% \meta{below} basically determine the $y$ coordinate of the text, while
% \meta{left} and \meta{right} determine the $x$ coordinate of the text. The
% remaining coordinate is determined using the \meta{position}.
%
% The \meta{text} is the actual text of the annotation.
%
% The \meta{x} and \meta{y} parameters are the actual coordinates where the
% arrow should point to. Note that \pkg{} will automatically insert a small
% distance between the arrow's tip and the coordinate given, such that the
% arrow is close to the coordinate but does not cover it.
%
% For example, the code
%
% |\draw[annotation left = {Atlas at 0.3}] to (0.11,0.4);|
%
% draws the text ``Atlas'' on the left-hand side of the image, at $y=0.3$. The
% arrow will point towards coordinate $(0.11, 0.4)$ but ends shortly before this
% coordinate such that the interesting feature to be labelled is not covered by
% the arrow's tip.
%
% \section{Configuring styles}
%
% \DescribeMacro{\flowchartset}
% Various options, like font size and so on, can be configured with the
% |\flowchartset| macro. It uses the key-value syntax from \tkz{}, e.g.:
%
% |\flowchartset{|\meta{key}| = |\meta{value}|, ...}|
%
% Multiple \meta{key} and \meta{value} pairs may be combined. The following
% sections list all possible configurations.
%
% |\flowchartset| can be put anywhere, but it makes sense to put it into the
% preamble of a document to ensure all images have the same look.
%
% There is also a default style. If no |\flowchartset| command is present, the
% default values for all options are taken. The default style used is as
% follows:
%
% \begin{verbatim}
% \flowchartset{
%   coarse grid color = red,
%   fine grid color = gray,
%   image label font = \sffamily\bfseries\small,
%   image label distance = 2mm,
%   image label back = black,
%   image label text = white,
%   coordinate label font = \sffamily\bfseries\scriptsize,
%   coordinate label distance = 2mm,
%   coordinate label back = black,
%   coordinate label text = white,
%   annotation font = \normalfont\small,
%   arrow distance = 1.5mm,
%   border thickness = 0.6pt,
%   arrow thickness = 0.4pt,
%   tip size = 1.2mm,
%   outer dist = 0.5cm,
% }
% \end{verbatim}
%
% The individual keys are described in the following sections.
%
% \subsection{Grid color}
%
% In most cases, it will not be necessary to adjust the grid colors. However,
% depending on the image, it may be desirable to do so. This is exactly what the
% \meta{coarse grid color} and \meta{fine grid color} are used for. Any color
% specification compatible to \tkz{} may be used. The defaults are red for the
% coarse grid and gray for the fine grid.
%
% \subsection{Font and color for the labels}
%
% The font for the image labels may be configured with \meta{image label font}.
% By default, the image labels are typeset with bold, small, sans-serif font.
%
% The background color of the image labels may be set using the
% \meta{image label back} kay, whereas the text color is specified with the
% \meta{image label text} key. Defaults for the background color and for the
% text color are black and white, respectively.
%
% \subsection{Distance of image labels to the image border}
%
% The \meta{image label distance} key configures the distance, $d$, from the
% image's border to the border of the image label, as shown in
% \autoref{fig:imagelabeldistance_example}. By default, it is set to 2\,mm.
%
% \begin{figure}
% \centering
% \flowchartset{
%    image label distance = 1cm}
% \begin{annotationimage}{width=5cm}{example-image}
%   \draw[image label = {Label at south east}];
%   \draw[>=latex, red, <->] (1,0.3) -- ++(-1cm,0) node[above, midway] {$d$};
%   \draw[>=latex, red, <->] (0.7,0) -- ++(0,1cm) node[midway, right] {$d$};
% \end{annotationimage}
% \caption{Illustration of the \meta{image label distance}}
% \label{fig:imagelabeldistance_example}
% \end{figure}
%
% \subsection{Font for annotations}
%
% The font used for annotations is set by \meta{annotation font}. By default,
% the |\normalfont| is used with small size.
%
% \subsection{Distance of arrow tips}
%
% As mentioned earlier, the arrows are shortened such that their tips don't
% cover the desired point. \autoref{fig:arrowdistance_example} illustrates this.
% All the arrows point to the same coordinate, $(0.5, 0.5)$, but they end at
% the distance $x$ away from the point. This distance may be configured using
% the \meta{arrow distance}. By default, this distance is set to 1.5\,mm. This
% ensures that the arrow tips are close enough to the interesting features, but
% not so close that they cover important parts of the image.
%
% \begin{figure}
% \centering
% \flowchartset{arrow distance = 1cm}
% \begin{annotationimage}{width=5cm}{example-image}
%   \draw[annotation right = {text at 0.5}] to (0.5,0.5);
%   \draw[annotation left = {text at 0.1}] to (0.5,0.5);
%   \draw[annotation left = {text at 0.7}] to (0.5,0.5);
%   \draw[annotation below = {text at 0.4}] to (0.5,0.5);
%   \draw[annotation above = {text at 0.8}] to (0.5,0.5);
%   \draw[red] (0.5,0.5) circle[radius = 1cm];
%   \draw[>=latex, red, <->] (0.5,0.5) -- ++(60:1cm) node[midway, left] {$x$};
% \end{annotationimage}
% \caption{Illustration of the \meta{arrow distance}}
% \label{fig:arrowdistance_example}
% \end{figure}
%
% \subsection{Thickness and size of the arrows}
%
% The arrows themselves consist of two parts: the inner part, which is the
% actual arrow, and the border, which is, by default, a white border around the
% arrow. The border is required to ensure that each arrow is visible, no matter
% on what background it is drawn. The thickness of the black line can be
% configured using the \meta{arrow thickness}, whose default value is 0.4\,pt.
% The thickness of the border around the arrow is configured with the
% \meta{border thickness}, having a default value of 0.6\,pt.
%
% The size of the round dot at the end of the arrows is configured using the
% \meta{tip size}. \autoref{fig:arrowthickness_tipsize_example} illustrates
% both, the \meta{arrow thickness}, and the \meta{tip size}, as parameters
% $a$ and $b$, respectively.
%
% \begin{figure}
% \centering
% \flowchartset{arrow thickness = 1cm, tip size = 2cm, arrow distance=0}
%   \let\tikzset\flowchartset
% \begin{tikzpicture}
%   \draw[annotation arrow] (0,0) -- (4cm,0);
%   \draw[>=latex, red, <->] (0.5,-0.5) -- ++(0,1cm) node[right, midway] {$a$};
%   \draw[>=latex, red, <->] (3cm,-1cm) -- ++(0,2cm) node[right, midway] {$b$};
% \end{tikzpicture}
% \caption{Illustration of the \meta{arrow thickness}, $a$, and
% the \meta{tip size}, $b$}
% \label{fig:arrowthickness_tipsize_example}
% \end{figure}
%
% \subsection{Distance of annotation texts from the image}
%
% The parameter \meta{outer dist} configures how far away from the image the
% annotation texts will be positioned. By default, this value is 0.5\,cm.
